%% Generated by Sphinx.
\def\sphinxdocclass{report}
\documentclass[letterpaper,10pt,english]{sphinxmanual}
\ifdefined\pdfpxdimen
   \let\sphinxpxdimen\pdfpxdimen\else\newdimen\sphinxpxdimen
\fi \sphinxpxdimen=.75bp\relax

\PassOptionsToPackage{warn}{textcomp}
\usepackage[utf8]{inputenc}
\ifdefined\DeclareUnicodeCharacter
% support both utf8 and utf8x syntaxes
  \ifdefined\DeclareUnicodeCharacterAsOptional
    \def\sphinxDUC#1{\DeclareUnicodeCharacter{"#1}}
  \else
    \let\sphinxDUC\DeclareUnicodeCharacter
  \fi
  \sphinxDUC{00A0}{\nobreakspace}
  \sphinxDUC{2500}{\sphinxunichar{2500}}
  \sphinxDUC{2502}{\sphinxunichar{2502}}
  \sphinxDUC{2514}{\sphinxunichar{2514}}
  \sphinxDUC{251C}{\sphinxunichar{251C}}
  \sphinxDUC{2572}{\textbackslash}
\fi
\usepackage{cmap}
\usepackage[T1]{fontenc}
\usepackage{amsmath,amssymb,amstext}
\usepackage{babel}



\usepackage{times}
\expandafter\ifx\csname T@LGR\endcsname\relax
\else
% LGR was declared as font encoding
  \substitutefont{LGR}{\rmdefault}{cmr}
  \substitutefont{LGR}{\sfdefault}{cmss}
  \substitutefont{LGR}{\ttdefault}{cmtt}
\fi
\expandafter\ifx\csname T@X2\endcsname\relax
  \expandafter\ifx\csname T@T2A\endcsname\relax
  \else
  % T2A was declared as font encoding
    \substitutefont{T2A}{\rmdefault}{cmr}
    \substitutefont{T2A}{\sfdefault}{cmss}
    \substitutefont{T2A}{\ttdefault}{cmtt}
  \fi
\else
% X2 was declared as font encoding
  \substitutefont{X2}{\rmdefault}{cmr}
  \substitutefont{X2}{\sfdefault}{cmss}
  \substitutefont{X2}{\ttdefault}{cmtt}
\fi


\usepackage[Bjarne]{fncychap}
\usepackage{sphinx}

\fvset{fontsize=\small}
\usepackage{geometry}


% Include hyperref last.
\usepackage{hyperref}
% Fix anchor placement for figures with captions.
\usepackage{hypcap}% it must be loaded after hyperref.
% Set up styles of URL: it should be placed after hyperref.
\urlstyle{same}
\addto\captionsenglish{\renewcommand{\contentsname}{Contents:}}

\usepackage{sphinxmessages}
\setcounter{tocdepth}{1}



\title{Oxford Endpoint Plotter}
\date{Jan 09, 2020}
\release{0.0.0}
\author{Michael Eller}
\newcommand{\sphinxlogo}{\vbox{}}
\renewcommand{\releasename}{Release}
\makeindex
\begin{document}

\pagestyle{empty}
\sphinxmaketitle
\pagestyle{plain}
\sphinxtableofcontents
\pagestyle{normal}
\phantomsection\label{\detokenize{index::doc}}



\chapter{Introduction}
\label{\detokenize{introduction:introduction}}\label{\detokenize{introduction::doc}}
This package is a Qt\sphinxhyphen{}based GUI for the UVML’s Oxford ICP/RIE tool. It’s primary purpose is to allow the user to plot the interferometric laser endpoint signal from the tool in real time, independent from the Oxford software. The main loop requests a value from an ADC over a serial connection and plots it in an embedded matplotlib animation. The user may also tweak settings such as sample rate, axes limits, averaging, etc.

A high\sphinxhyphen{}level diagram showing how the program works is given below:

\noindent\sphinxincludegraphics{{diagram}.png}


\chapter{Manual}
\label{\detokenize{manual:manual}}\label{\detokenize{manual::doc}}
Once you have started the program, the main window will open and should look like this:

\noindent\sphinxincludegraphics{{mainwindow}.png}

To open the plot, all you need to do is click the big ‘Plot’ button. Once open, the animation will look something like this:

\noindent\sphinxincludegraphics{{plot}.png}

You can save the current plot picture as an image, or save all recorded data to a csv file. This will only record the raw data (not the filtered data or the derivative); however, you can calculate those easily from the raw data. If you get the following warning message:

\noindent\sphinxincludegraphics{{warning}.png}

You will need to set the serial port before you can do anything else. This is to stop the program from running without a valid serial port to connect to. Note: If the program crashes, the most likely cause is the serial port connection. Double check that it exists and there aren’t any active connections to it already. To set the serial port, click Menu \textgreater{} Settings \textgreater{} Port Connection. The following dialog will open:

\noindent\sphinxincludegraphics{{port}.png}

Select the appropriate serial port. In this case it’s going to be whichever port shows the embedded arduino microcontroller. To clear your selection, click on the ‘Reset’ button. If you wish to rescan the system’s available serial ports, click ‘Scan’. To save your selection, click on the save button. The warning dialog will no longer appear.

If you want to display the raw voltage without the plot, you can hit the ‘Enable’ checkbox on the main menu. The program will then poll the ADC and update the LCD every 1s. The live plot and LCD can not run at the same time. We are limited to one serial connection to the ADC at a time. Be patient when you click the checkbox, the program halts for 2s to make sure the serial connection is initialized and ready.

\noindent\sphinxincludegraphics{{mainwindow_lcd}.png}

While the plot is not open, you may alter any of the settings from the menu. They should be pretty self\sphinxhyphen{}explanatory. If the user inputs an invalid value, nothing will happen.

If the plot is open, the user may still alter a few settings dynamically: the window filter and the axes limits. To do this, move the plot window so you can see both the plot and the main window. Then alter the settings you wish to change and save.

\noindent\sphinxincludegraphics{{dynamic_settings}.png}


\chapter{EndpointPlotter}
\label{\detokenize{modules:endpointplotter}}\label{\detokenize{modules::doc}}

\section{mainWindow module}
\label{\detokenize{mainWindow:module-mainWindow}}\label{\detokenize{mainWindow:mainwindow-module}}\label{\detokenize{mainWindow::doc}}\index{mainWindow (module)@\spxentry{mainWindow}\spxextra{module}}\begin{quote}\begin{description}
\item[{platform}] \leavevmode
Unix, Windows

\item[{synopsis}] \leavevmode
This is the main module. Run this file to start the GUI.

\item[{moduleauthor}] \leavevmode
Michael Eller \textless{}\sphinxhref{mailto:mbe9a@virginia.edu}{mbe9a@virginia.edu}\textgreater{}

\end{description}\end{quote}
\index{Ui\_MainWindow (class in mainWindow)@\spxentry{Ui\_MainWindow}\spxextra{class in mainWindow}}

\begin{fulllineitems}
\phantomsection\label{\detokenize{mainWindow:mainWindow.Ui_MainWindow}}\pysigline{\sphinxbfcode{\sphinxupquote{class }}\sphinxcode{\sphinxupquote{mainWindow.}}\sphinxbfcode{\sphinxupquote{Ui\_MainWindow}}}
Bases: \sphinxcode{\sphinxupquote{object}}

The graphical structure of \sphinxstyleemphasis{Ui\_MainWindow} was generated by QtDesigner. The window contains a settings menu,
LCD display, and a button to launch a live plot.
\index{setupUi() (mainWindow.Ui\_MainWindow method)@\spxentry{setupUi()}\spxextra{mainWindow.Ui\_MainWindow method}}

\begin{fulllineitems}
\phantomsection\label{\detokenize{mainWindow:mainWindow.Ui_MainWindow.setupUi}}\pysiglinewithargsret{\sphinxbfcode{\sphinxupquote{setupUi}}}{\emph{MainWindow}}{}
This function initializes the window by altering the \sphinxstyleemphasis{MainWindow} object passed (by reference) to it.
\begin{quote}\begin{description}
\item[{Parameters}] \leavevmode
\sphinxstyleliteralstrong{\sphinxupquote{MainWindow}} \textendash{} this must be of type PyQt5.QtWidgets.QMainWindow

\end{description}\end{quote}

\end{fulllineitems}

\index{retranslateUi() (mainWindow.Ui\_MainWindow method)@\spxentry{retranslateUi()}\spxextra{mainWindow.Ui\_MainWindow method}}

\begin{fulllineitems}
\phantomsection\label{\detokenize{mainWindow:mainWindow.Ui_MainWindow.retranslateUi}}\pysiglinewithargsret{\sphinxbfcode{\sphinxupquote{retranslateUi}}}{\emph{MainWindow}}{}
This function translates the GUI objects of this class to the passed \sphinxstyleemphasis{MainWindow} objcect,
then sets the text of the various objects in the main window.
\begin{quote}\begin{description}
\item[{Parameters}] \leavevmode
\sphinxstyleliteralstrong{\sphinxupquote{MainWindow}} \textendash{} this must be of type PyQt5.QtWidgets.QMainWindow

\end{description}\end{quote}

\end{fulllineitems}

\index{check\_port() (mainWindow.Ui\_MainWindow method)@\spxentry{check\_port()}\spxextra{mainWindow.Ui\_MainWindow method}}

\begin{fulllineitems}
\phantomsection\label{\detokenize{mainWindow:mainWindow.Ui_MainWindow.check_port}}\pysiglinewithargsret{\sphinxbfcode{\sphinxupquote{check\_port}}}{}{}
Checks to see if the user has set a port. If not, the user will not be able to start the live plot or LCD.

\end{fulllineitems}

\index{open\_settings\_filter() (mainWindow.Ui\_MainWindow method)@\spxentry{open\_settings\_filter()}\spxextra{mainWindow.Ui\_MainWindow method}}

\begin{fulllineitems}
\phantomsection\label{\detokenize{mainWindow:mainWindow.Ui_MainWindow.open_settings_filter}}\pysiglinewithargsret{\sphinxbfcode{\sphinxupquote{open\_settings\_filter}}}{}{}
This function spawns an instance of \sphinxstyleemphasis{windowFilterGUI.Ui\_Dialog}. This is triggered by
clicking Menu \textgreater{} Settings \textgreater{} Window Filter.

\end{fulllineitems}

\index{open\_settings\_sample\_rate() (mainWindow.Ui\_MainWindow method)@\spxentry{open\_settings\_sample\_rate()}\spxextra{mainWindow.Ui\_MainWindow method}}

\begin{fulllineitems}
\phantomsection\label{\detokenize{mainWindow:mainWindow.Ui_MainWindow.open_settings_sample_rate}}\pysiglinewithargsret{\sphinxbfcode{\sphinxupquote{open\_settings\_sample\_rate}}}{}{}
This function spawns an instance of \sphinxstyleemphasis{sampRateGUI.Ui\_Dialog}. This is triggered by
clicking Menu \textgreater{} Settings \textgreater{} Sample Rate.

\end{fulllineitems}

\index{open\_settings\_scale\_axes() (mainWindow.Ui\_MainWindow method)@\spxentry{open\_settings\_scale\_axes()}\spxextra{mainWindow.Ui\_MainWindow method}}

\begin{fulllineitems}
\phantomsection\label{\detokenize{mainWindow:mainWindow.Ui_MainWindow.open_settings_scale_axes}}\pysiglinewithargsret{\sphinxbfcode{\sphinxupquote{open\_settings\_scale\_axes}}}{}{}
This function spawns an instance of \sphinxstyleemphasis{scaleAxesGUI.Ui\_Dialog}. This is triggered by
clicking Menu \textgreater{} Settings \textgreater{} Scale Axes.

\end{fulllineitems}

\index{open\_settings\_port() (mainWindow.Ui\_MainWindow method)@\spxentry{open\_settings\_port()}\spxextra{mainWindow.Ui\_MainWindow method}}

\begin{fulllineitems}
\phantomsection\label{\detokenize{mainWindow:mainWindow.Ui_MainWindow.open_settings_port}}\pysiglinewithargsret{\sphinxbfcode{\sphinxupquote{open\_settings\_port}}}{}{}
This function spawns an instance of \sphinxstyleemphasis{portsGUI.Ui\_Dialog}. This is triggered by
clicking Menu \textgreater{} Settings \textgreater{} Port Connection.

\end{fulllineitems}

\index{show\_plot() (mainWindow.Ui\_MainWindow method)@\spxentry{show\_plot()}\spxextra{mainWindow.Ui\_MainWindow method}}

\begin{fulllineitems}
\phantomsection\label{\detokenize{mainWindow:mainWindow.Ui_MainWindow.show_plot}}\pysiglinewithargsret{\sphinxbfcode{\sphinxupquote{show\_plot}}}{}{}
This function spawns an instance of \sphinxstyleemphasis{plotGUI.Ui\_Dialog}. It is triggered by
the user clicking the large ‘Plot’ button.

\end{fulllineitems}

\index{toggle\_timer() (mainWindow.Ui\_MainWindow method)@\spxentry{toggle\_timer()}\spxextra{mainWindow.Ui\_MainWindow method}}

\begin{fulllineitems}
\phantomsection\label{\detokenize{mainWindow:mainWindow.Ui_MainWindow.toggle_timer}}\pysiglinewithargsret{\sphinxbfcode{\sphinxupquote{toggle\_timer}}}{}{}
This function is triggered on every click of the check box. If the check box is enabled and
the user has set the serial port, this function enables a timer that will update the LCD every 1s.

\end{fulllineitems}

\index{update\_display() (mainWindow.Ui\_MainWindow method)@\spxentry{update\_display()}\spxextra{mainWindow.Ui\_MainWindow method}}

\begin{fulllineitems}
\phantomsection\label{\detokenize{mainWindow:mainWindow.Ui_MainWindow.update_display}}\pysiglinewithargsret{\sphinxbfcode{\sphinxupquote{update\_display}}}{}{}
This function will execute on the 1s timer’s timeout. It will then request the most recent ADC value
from the serial port and display the corrected value on the LCD.

\end{fulllineitems}


\end{fulllineitems}



\subsection{Usage}
\label{\detokenize{mainWindow:usage}}
\begin{sphinxVerbatim}[commandchars=\\\{\}]
\PYG{k+kn}{import} \PYG{n+nn}{sys}
\PYG{k+kn}{from} \PYG{n+nn}{PyQt5} \PYG{k+kn}{import} \PYG{n}{QtWidgets}
\PYG{k+kn}{from} \PYG{n+nn}{mainWindow} \PYG{k+kn}{import} \PYG{n}{Ui\PYGZus{}MainWindow}

\PYG{n}{app} \PYG{o}{=} \PYG{n}{QtWidgets}\PYG{o}{.}\PYG{n}{QApplication}\PYG{p}{(}\PYG{n}{sys}\PYG{o}{.}\PYG{n}{argv}\PYG{p}{)}
\PYG{n}{MainWindow} \PYG{o}{=} \PYG{n}{QtWidgets}\PYG{o}{.}\PYG{n}{QMainWindow}\PYG{p}{(}\PYG{p}{)}
\PYG{n}{ui} \PYG{o}{=} \PYG{n}{Ui\PYGZus{}MainWindow}\PYG{p}{(}\PYG{p}{)}
\PYG{n}{ui}\PYG{o}{.}\PYG{n}{setupUi}\PYG{p}{(}\PYG{n}{MainWindow}\PYG{p}{)}
\PYG{n}{MainWindow}\PYG{o}{.}\PYG{n}{show}\PYG{p}{(}\PYG{p}{)}
\PYG{n}{sys}\PYG{o}{.}\PYG{n}{exit}\PYG{p}{(}\PYG{n}{app}\PYG{o}{.}\PYG{n}{exec\PYGZus{}}\PYG{p}{(}\PYG{p}{)}\PYG{p}{)}
\end{sphinxVerbatim}


\section{plotGUI module}
\label{\detokenize{plotGUI:module-plotGUI}}\label{\detokenize{plotGUI:plotgui-module}}\label{\detokenize{plotGUI::doc}}\index{plotGUI (module)@\spxentry{plotGUI}\spxextra{module}}\begin{quote}\begin{description}
\item[{platform}] \leavevmode
Unix, Windows

\item[{synopsis}] \leavevmode
This module contains generated code from QtDesigner and custom code for the GUI containing the live plot.

\item[{moduleauthor}] \leavevmode
Michael Eller \textless{}\sphinxhref{mailto:mbe9a@virginia.edu}{mbe9a@virginia.edu}\textgreater{}

\end{description}\end{quote}
\index{stop (in module plotGUI)@\spxentry{stop}\spxextra{in module plotGUI}}

\begin{fulllineitems}
\phantomsection\label{\detokenize{plotGUI:plotGUI.stop}}\pysigline{\sphinxcode{\sphinxupquote{plotGUI.}}\sphinxbfcode{\sphinxupquote{stop}}\sphinxbfcode{\sphinxupquote{ = 0}}}
This global member is used to create a soft stop signal for a plotGUI.Ui\_Dialog object

\end{fulllineitems}

\index{initial\_hash (in module plotGUI)@\spxentry{initial\_hash}\spxextra{in module plotGUI}}

\begin{fulllineitems}
\phantomsection\label{\detokenize{plotGUI:plotGUI.initial_hash}}\pysigline{\sphinxcode{\sphinxupquote{plotGUI.}}\sphinxbfcode{\sphinxupquote{initial\_hash}}\sphinxbfcode{\sphinxupquote{ = 0}}}
\sphinxstyleemphasis{initial\_hash} is set on the creation of a plotGUI.Ui\_Dialog object.
Hashing is used to detect a change in the settings files.
The program will periodically hash the settings files and compare with initial\_hash.
If a change has been detected, plotGUI.Ui\_dialog.update\_plot() will be called.

\end{fulllineitems}

\index{MyThread (class in plotGUI)@\spxentry{MyThread}\spxextra{class in plotGUI}}

\begin{fulllineitems}
\phantomsection\label{\detokenize{plotGUI:plotGUI.MyThread}}\pysiglinewithargsret{\sphinxbfcode{\sphinxupquote{class }}\sphinxcode{\sphinxupquote{plotGUI.}}\sphinxbfcode{\sphinxupquote{MyThread}}}{\emph{callback1}, \emph{callback2}}{}
Bases: \sphinxcode{\sphinxupquote{PyQt5.QtCore.QThread}}

This is a simple class to implement a QThread object and specify its \sphinxstyleemphasis{run()} method.
\index{\_\_init\_\_() (plotGUI.MyThread method)@\spxentry{\_\_init\_\_()}\spxextra{plotGUI.MyThread method}}

\begin{fulllineitems}
\phantomsection\label{\detokenize{plotGUI:plotGUI.MyThread.__init__}}\pysiglinewithargsret{\sphinxbfcode{\sphinxupquote{\_\_init\_\_}}}{\emph{callback1}, \emph{callback2}}{}
Initialize the QThread and set the callback stubs.
\begin{quote}\begin{description}
\item[{Parameters}] \leavevmode\begin{itemize}
\item {} 
\sphinxstyleliteralstrong{\sphinxupquote{callback1}} \textendash{} Callback function for the thread to execute within \sphinxstyleemphasis{dataSendLoop()}. T
his callback must send a new data point to the plot.

\item {} 
\sphinxstyleliteralstrong{\sphinxupquote{callback2}} \textendash{} Callback function triggered in \sphinxstyleemphasis{dataSendLoop()}
and is responsible for triggering \sphinxstyleemphasis{update\_plot()}.

\end{itemize}

\end{description}\end{quote}

\end{fulllineitems}

\index{run() (plotGUI.MyThread method)@\spxentry{run()}\spxextra{plotGUI.MyThread method}}

\begin{fulllineitems}
\phantomsection\label{\detokenize{plotGUI:plotGUI.MyThread.run}}\pysiglinewithargsret{\sphinxbfcode{\sphinxupquote{run}}}{}{}
This overrides the parent \sphinxstyleemphasis{run()} method. Thread will run the infinite loop in \sphinxstyleemphasis{dataSendLoop()}.

\end{fulllineitems}


\end{fulllineitems}

\index{Ui\_Dialog (class in plotGUI)@\spxentry{Ui\_Dialog}\spxextra{class in plotGUI}}

\begin{fulllineitems}
\phantomsection\label{\detokenize{plotGUI:plotGUI.Ui_Dialog}}\pysigline{\sphinxbfcode{\sphinxupquote{class }}\sphinxcode{\sphinxupquote{plotGUI.}}\sphinxbfcode{\sphinxupquote{Ui\_Dialog}}}
Bases: \sphinxcode{\sphinxupquote{object}}

The graphical structure of this \sphinxstyleemphasis{Ui\_Dialog} was generated by QtDesigner.
The window contains a few buttons and an embedded matplotlib animation.
\index{setupUi() (plotGUI.Ui\_Dialog method)@\spxentry{setupUi()}\spxextra{plotGUI.Ui\_Dialog method}}

\begin{fulllineitems}
\phantomsection\label{\detokenize{plotGUI:plotGUI.Ui_Dialog.setupUi}}\pysiglinewithargsret{\sphinxbfcode{\sphinxupquote{setupUi}}}{\emph{Dialog}}{}
This function initializes the window by altering the \sphinxstyleemphasis{Dialog} object passed (by reference) to it.
\begin{quote}\begin{description}
\item[{Parameters}] \leavevmode
\sphinxstyleliteralstrong{\sphinxupquote{Dialog}} \textendash{} This must be of type PyQt5.QtWidgets.QDialog.

\end{description}\end{quote}

\end{fulllineitems}

\index{retranslateUi() (plotGUI.Ui\_Dialog method)@\spxentry{retranslateUi()}\spxextra{plotGUI.Ui\_Dialog method}}

\begin{fulllineitems}
\phantomsection\label{\detokenize{plotGUI:plotGUI.Ui_Dialog.retranslateUi}}\pysiglinewithargsret{\sphinxbfcode{\sphinxupquote{retranslateUi}}}{\emph{Dialog}}{}
This function translates the GUI objects of this class to the passed \sphinxstyleemphasis{Dialog} objcect,
then sets the text of the various objects in the window.
\begin{quote}\begin{description}
\item[{Parameters}] \leavevmode
\sphinxstyleliteralstrong{\sphinxupquote{Dialog}} \textendash{} This must be of type PyQt5.QtWidgets.QDialog.

\end{description}\end{quote}

\end{fulllineitems}

\index{addData\_callbackFunc() (plotGUI.Ui\_Dialog method)@\spxentry{addData\_callbackFunc()}\spxextra{plotGUI.Ui\_Dialog method}}

\begin{fulllineitems}
\phantomsection\label{\detokenize{plotGUI:plotGUI.Ui_Dialog.addData_callbackFunc}}\pysiglinewithargsret{\sphinxbfcode{\sphinxupquote{addData\_callbackFunc}}}{\emph{value}}{}
This function adds the most recent ADC value to the buffer in the animation object.
\begin{quote}\begin{description}
\item[{Parameters}] \leavevmode
\sphinxstyleliteralstrong{\sphinxupquote{value}} \textendash{} number from the serial port. This is ‘emitted’ from \sphinxstyleemphasis{mySrc} in \sphinxstyleemphasis{dataSendLoot()}.

\end{description}\end{quote}

\end{fulllineitems}

\index{update\_plot() (plotGUI.Ui\_Dialog method)@\spxentry{update\_plot()}\spxextra{plotGUI.Ui\_Dialog method}}

\begin{fulllineitems}
\phantomsection\label{\detokenize{plotGUI:plotGUI.Ui_Dialog.update_plot}}\pysiglinewithargsret{\sphinxbfcode{\sphinxupquote{update\_plot}}}{\emph{value}}{}
This function calls a series of functions within the animation object to update the figure axes and filtering.
\begin{quote}\begin{description}
\item[{Parameters}] \leavevmode
\sphinxstyleliteralstrong{\sphinxupquote{value}} \textendash{} the format of this function requires \sphinxstyleemphasis{value} to be present. It is unused in this function.

\end{description}\end{quote}

\end{fulllineitems}

\index{close() (plotGUI.Ui\_Dialog method)@\spxentry{close()}\spxextra{plotGUI.Ui\_Dialog method}}

\begin{fulllineitems}
\phantomsection\label{\detokenize{plotGUI:plotGUI.Ui_Dialog.close}}\pysiglinewithargsret{\sphinxbfcode{\sphinxupquote{close}}}{}{}
This function creates a controlled and soft close.
The figure and serial must be closed before the window is allowed to close.

\end{fulllineitems}

\index{save\_image() (plotGUI.Ui\_Dialog method)@\spxentry{save\_image()}\spxextra{plotGUI.Ui\_Dialog method}}

\begin{fulllineitems}
\phantomsection\label{\detokenize{plotGUI:plotGUI.Ui_Dialog.save_image}}\pysiglinewithargsret{\sphinxbfcode{\sphinxupquote{save\_image}}}{\emph{Dialog}}{}
Saves the current plot canvas as an image file.
It will open a save dialog in order to get the desired file path from the user.
\begin{quote}\begin{description}
\item[{Parameters}] \leavevmode
\sphinxstyleliteralstrong{\sphinxupquote{Dialog}} \textendash{} The same PyQt5.QtWidgets.QDialog object passed to the PlotGUI.Ui\_Dialog.

\end{description}\end{quote}

\end{fulllineitems}

\index{save\_csv() (plotGUI.Ui\_Dialog method)@\spxentry{save\_csv()}\spxextra{plotGUI.Ui\_Dialog method}}

\begin{fulllineitems}
\phantomsection\label{\detokenize{plotGUI:plotGUI.Ui_Dialog.save_csv}}\pysiglinewithargsret{\sphinxbfcode{\sphinxupquote{save\_csv}}}{\emph{Dialog}}{}
This function saves all recorded data in a simple format.
Opens a save file dialog to get the desired file path from the user.
\begin{quote}\begin{description}
\item[{Parameters}] \leavevmode
\sphinxstyleliteralstrong{\sphinxupquote{Dialog}} \textendash{} The same PyQt5.QtWidgets.QDialog object passed to the PlotGUI.Ui\_Dialog.

\end{description}\end{quote}

\end{fulllineitems}


\end{fulllineitems}

\index{Communicate (class in plotGUI)@\spxentry{Communicate}\spxextra{class in plotGUI}}

\begin{fulllineitems}
\phantomsection\label{\detokenize{plotGUI:plotGUI.Communicate}}\pysigline{\sphinxbfcode{\sphinxupquote{class }}\sphinxcode{\sphinxupquote{plotGUI.}}\sphinxbfcode{\sphinxupquote{Communicate}}}
Bases: \sphinxcode{\sphinxupquote{PyQt5.QtCore.QObject}}

In order to cleanly send data throughout the different processes within the GUI,
a thread\sphinxhyphen{}safe mechanism must be used. This is an extension of a built\sphinxhyphen{}in pyqt signal.
It can be connected to a slot (function) that will trigger when the signal emits a value.
\index{data\_signal (plotGUI.Communicate attribute)@\spxentry{data\_signal}\spxextra{plotGUI.Communicate attribute}}

\begin{fulllineitems}
\phantomsection\label{\detokenize{plotGUI:plotGUI.Communicate.data_signal}}\pysigline{\sphinxbfcode{\sphinxupquote{data\_signal}}}
\end{fulllineitems}


\end{fulllineitems}

\index{dataSendLoop() (in module plotGUI)@\spxentry{dataSendLoop()}\spxextra{in module plotGUI}}

\begin{fulllineitems}
\phantomsection\label{\detokenize{plotGUI:plotGUI.dataSendLoop}}\pysiglinewithargsret{\sphinxcode{\sphinxupquote{plotGUI.}}\sphinxbfcode{\sphinxupquote{dataSendLoop}}}{\emph{addData\_callbackFunc}, \emph{update\_plot}}{}
This is the infinite loop executed by a separate thread. Based on the sample rate set by the user,
it will periodically poll a new value over the serial port and send it to the animation through a
signal\sphinxhyphen{}slot mechanism. This function is also responsible for re\sphinxhyphen{}hashing the settings files and
checking for necessary updates to the plot axes and filtering. If a change is detected, it will
trigger the \sphinxstyleemphasis{update\_plot()} function.
\begin{quote}\begin{description}
\item[{Parameters}] \leavevmode\begin{itemize}
\item {} 
\sphinxstyleliteralstrong{\sphinxupquote{addData\_callbackFunc}} \textendash{} slot function for adding data to the plot.

\item {} 
\sphinxstyleliteralstrong{\sphinxupquote{update\_plot}} \textendash{} slot function for updating the plot filtering and axes.

\end{itemize}

\end{description}\end{quote}

\end{fulllineitems}

\index{get\_interval() (in module plotGUI)@\spxentry{get\_interval()}\spxextra{in module plotGUI}}

\begin{fulllineitems}
\phantomsection\label{\detokenize{plotGUI:plotGUI.get_interval}}\pysiglinewithargsret{\sphinxcode{\sphinxupquote{plotGUI.}}\sphinxbfcode{\sphinxupquote{get\_interval}}}{}{}
This is a helper function to calculate the time between samples.
It reads the sample rate value from the settings file and takes the inverse.
\begin{quote}\begin{description}
\item[{Returns}] \leavevmode
the interval in seconds.

\end{description}\end{quote}

\end{fulllineitems}

\index{check\_for\_change() (in module plotGUI)@\spxentry{check\_for\_change()}\spxextra{in module plotGUI}}

\begin{fulllineitems}
\phantomsection\label{\detokenize{plotGUI:plotGUI.check_for_change}}\pysiglinewithargsret{\sphinxcode{\sphinxupquote{plotGUI.}}\sphinxbfcode{\sphinxupquote{check\_for\_change}}}{}{}
If a change was detected return True. If no change, return False.
\begin{quote}\begin{description}
\item[{Returns}] \leavevmode
bool

\end{description}\end{quote}

\end{fulllineitems}

\index{get\_hash() (in module plotGUI)@\spxentry{get\_hash()}\spxextra{in module plotGUI}}

\begin{fulllineitems}
\phantomsection\label{\detokenize{plotGUI:plotGUI.get_hash}}\pysiglinewithargsret{\sphinxcode{\sphinxupquote{plotGUI.}}\sphinxbfcode{\sphinxupquote{get\_hash}}}{}{}
This function uses \sphinxstyleemphasis{hashlib} to generate an md5 hash of the settings file.
\begin{quote}\begin{description}
\item[{Returns}] \leavevmode
the hash of the settings file

\end{description}\end{quote}

\end{fulllineitems}



\section{animation module}
\label{\detokenize{animation:module-animation}}\label{\detokenize{animation:animation-module}}\label{\detokenize{animation::doc}}\index{animation (module)@\spxentry{animation}\spxextra{module}}\begin{quote}\begin{description}
\item[{platform}] \leavevmode
Unix, Windows

\item[{synopsis}] \leavevmode
This module contains the endpoint plot code. This code was adapted in part from K. Mulier’s
example on stackoverflow.

\item[{moduleauthor}] \leavevmode
Michael Eller \textless{}\sphinxhref{mailto:mbe9a@virginia.edu}{mbe9a@virginia.edu}\textgreater{}

\end{description}\end{quote}
\index{CustomFigCanvas (class in animation)@\spxentry{CustomFigCanvas}\spxextra{class in animation}}

\begin{fulllineitems}
\phantomsection\label{\detokenize{animation:animation.CustomFigCanvas}}\pysigline{\sphinxbfcode{\sphinxupquote{class }}\sphinxcode{\sphinxupquote{animation.}}\sphinxbfcode{\sphinxupquote{CustomFigCanvas}}}
Bases: \sphinxcode{\sphinxupquote{matplotlib.backends.backend\_qt5agg.FigureCanvasQTAgg}}, \sphinxcode{\sphinxupquote{matplotlib.animation.TimedAnimation}}

CustomFigCanvas is a class designed to allow integration of a matplotlib animation into a Qt backend.
The animation will display the recorded voltage from the ADC and its corresponding differential derivative.
The user can edit the axes limits and change the window filtering dynamically from the main menu.
\index{\_\_init\_\_() (animation.CustomFigCanvas method)@\spxentry{\_\_init\_\_()}\spxextra{animation.CustomFigCanvas method}}

\begin{fulllineitems}
\phantomsection\label{\detokenize{animation:animation.CustomFigCanvas.__init__}}\pysiglinewithargsret{\sphinxbfcode{\sphinxupquote{\_\_init\_\_}}}{}{}
Initializes the object and its parent objects. Creates the figure, axes, and artists to be displayed.

\end{fulllineitems}

\index{get\_windowed\_value() (animation.CustomFigCanvas method)@\spxentry{get\_windowed\_value()}\spxextra{animation.CustomFigCanvas method}}

\begin{fulllineitems}
\phantomsection\label{\detokenize{animation:animation.CustomFigCanvas.get_windowed_value}}\pysiglinewithargsret{\sphinxbfcode{\sphinxupquote{get\_windowed\_value}}}{\emph{window}}{}
Helper function that simply calculates the average of all the samples in the window.
\begin{quote}\begin{description}
\item[{Parameters}] \leavevmode
\sphinxstyleliteralstrong{\sphinxupquote{window}} \textendash{} the list containing the data to be averaged

\item[{Returns}] \leavevmode
the average

\end{description}\end{quote}

\end{fulllineitems}

\index{new\_frame\_seq() (animation.CustomFigCanvas method)@\spxentry{new\_frame\_seq()}\spxextra{animation.CustomFigCanvas method}}

\begin{fulllineitems}
\phantomsection\label{\detokenize{animation:animation.CustomFigCanvas.new_frame_seq}}\pysiglinewithargsret{\sphinxbfcode{\sphinxupquote{new\_frame\_seq}}}{}{}
Return a new sequence of frame information.

\end{fulllineitems}

\index{addData() (animation.CustomFigCanvas method)@\spxentry{addData()}\spxextra{animation.CustomFigCanvas method}}

\begin{fulllineitems}
\phantomsection\label{\detokenize{animation:animation.CustomFigCanvas.addData}}\pysiglinewithargsret{\sphinxbfcode{\sphinxupquote{addData}}}{\emph{value}}{}
This function is called by the signal\sphinxhyphen{}slot mechanism within the plotGUI Dialog.
It adds a value to the plots received data buffer, \sphinxstyleemphasis{addedData}.
\begin{quote}\begin{description}
\item[{Parameters}] \leavevmode
\sphinxstyleliteralstrong{\sphinxupquote{value}} \textendash{} received value from the serial connection to the ADC

\end{description}\end{quote}

\end{fulllineitems}

\index{close() (animation.CustomFigCanvas method)@\spxentry{close()}\spxextra{animation.CustomFigCanvas method}}

\begin{fulllineitems}
\phantomsection\label{\detokenize{animation:animation.CustomFigCanvas.close}}\pysiglinewithargsret{\sphinxbfcode{\sphinxupquote{close}}}{}{}
Soft close function. Closes the figure contained in the object.

\end{fulllineitems}

\index{update\_xlim() (animation.CustomFigCanvas method)@\spxentry{update\_xlim()}\spxextra{animation.CustomFigCanvas method}}

\begin{fulllineitems}
\phantomsection\label{\detokenize{animation:animation.CustomFigCanvas.update_xlim}}\pysiglinewithargsret{\sphinxbfcode{\sphinxupquote{update\_xlim}}}{}{}
This function handles updating the x\sphinxhyphen{}axis range dynamically.

\end{fulllineitems}

\index{generate\_xticklabels() (animation.CustomFigCanvas method)@\spxentry{generate\_xticklabels()}\spxextra{animation.CustomFigCanvas method}}

\begin{fulllineitems}
\phantomsection\label{\detokenize{animation:animation.CustomFigCanvas.generate_xticklabels}}\pysiglinewithargsret{\sphinxbfcode{\sphinxupquote{generate\_xticklabels}}}{}{}
Since the animation progresses from right to left, the left\sphinxhyphen{}most point is actually in the past,
i.e. negative time. The simples way to do this is to keep the axis limits from 0 to xlim and alter the labels.

\end{fulllineitems}

\index{update\_ylim() (animation.CustomFigCanvas method)@\spxentry{update\_ylim()}\spxextra{animation.CustomFigCanvas method}}

\begin{fulllineitems}
\phantomsection\label{\detokenize{animation:animation.CustomFigCanvas.update_ylim}}\pysiglinewithargsret{\sphinxbfcode{\sphinxupquote{update\_ylim}}}{}{}
Update the data y\sphinxhyphen{}axis limits according to the input from the user.
The derivative limits are hard\sphinxhyphen{}coded based on the ymax of the data axis.

\end{fulllineitems}

\index{update\_window() (animation.CustomFigCanvas method)@\spxentry{update\_window()}\spxextra{animation.CustomFigCanvas method}}

\begin{fulllineitems}
\phantomsection\label{\detokenize{animation:animation.CustomFigCanvas.update_window}}\pysiglinewithargsret{\sphinxbfcode{\sphinxupquote{update\_window}}}{}{}
This function updates the \sphinxstyleemphasis{windows\_samples} variable from the settings file.

\end{fulllineitems}


\end{fulllineitems}



\section{portsGUI module}
\label{\detokenize{portsGUI:module-portsGUI}}\label{\detokenize{portsGUI:portsgui-module}}\label{\detokenize{portsGUI::doc}}\index{portsGUI (module)@\spxentry{portsGUI}\spxextra{module}}\begin{quote}\begin{description}
\item[{platform}] \leavevmode
Windows

\item[{synopsis}] \leavevmode
This module is mostly generated code from QtDesigner. It describes the format of the settings
dialog for editing the port of the ADC.

\item[{moduleauthor}] \leavevmode
Michael Eller \textless{}\sphinxhref{mailto:mbe9a@virginia.edu}{mbe9a@virginia.edu}\textgreater{}

\end{description}\end{quote}
\index{Ui\_Dialog (class in portsGUI)@\spxentry{Ui\_Dialog}\spxextra{class in portsGUI}}

\begin{fulllineitems}
\phantomsection\label{\detokenize{portsGUI:portsGUI.Ui_Dialog}}\pysigline{\sphinxbfcode{\sphinxupquote{class }}\sphinxcode{\sphinxupquote{portsGUI.}}\sphinxbfcode{\sphinxupquote{Ui\_Dialog}}}
Bases: \sphinxcode{\sphinxupquote{object}}

The graphical structure of this \sphinxstyleemphasis{Ui\_Dialog} was generated by QtDesigner.
The window allows the user to view available serial ports and choose one.
\index{setupUi() (portsGUI.Ui\_Dialog method)@\spxentry{setupUi()}\spxextra{portsGUI.Ui\_Dialog method}}

\begin{fulllineitems}
\phantomsection\label{\detokenize{portsGUI:portsGUI.Ui_Dialog.setupUi}}\pysiglinewithargsret{\sphinxbfcode{\sphinxupquote{setupUi}}}{\emph{Dialog}}{}
This function initializes the window by altering the \sphinxstyleemphasis{Dialog} object passed (by reference) to it.
\begin{quote}\begin{description}
\item[{Parameters}] \leavevmode
\sphinxstyleliteralstrong{\sphinxupquote{Dialog}} \textendash{} This must be of type PyQt5.QtWidgets.QDialog.

\end{description}\end{quote}

\end{fulllineitems}

\index{retranslateUi() (portsGUI.Ui\_Dialog method)@\spxentry{retranslateUi()}\spxextra{portsGUI.Ui\_Dialog method}}

\begin{fulllineitems}
\phantomsection\label{\detokenize{portsGUI:portsGUI.Ui_Dialog.retranslateUi}}\pysiglinewithargsret{\sphinxbfcode{\sphinxupquote{retranslateUi}}}{\emph{Dialog}}{}
This function translates the GUI objects of this class to the passed \sphinxstyleemphasis{Dialog} objcect,
then sets the text of the various objects in the window.
\begin{quote}\begin{description}
\item[{Parameters}] \leavevmode
\sphinxstyleliteralstrong{\sphinxupquote{Dialog}} \textendash{} This must be of type PyQt5.QtWidgets.QDialog.

\end{description}\end{quote}

\end{fulllineitems}

\index{highlight\_port() (portsGUI.Ui\_Dialog method)@\spxentry{highlight\_port()}\spxextra{portsGUI.Ui\_Dialog method}}

\begin{fulllineitems}
\phantomsection\label{\detokenize{portsGUI:portsGUI.Ui_Dialog.highlight_port}}\pysiglinewithargsret{\sphinxbfcode{\sphinxupquote{highlight\_port}}}{}{}
This function will highlight the saved serial port if it still exists in the list.

\end{fulllineitems}

\index{populate\_list() (portsGUI.Ui\_Dialog method)@\spxentry{populate\_list()}\spxextra{portsGUI.Ui\_Dialog method}}

\begin{fulllineitems}
\phantomsection\label{\detokenize{portsGUI:portsGUI.Ui_Dialog.populate_list}}\pysiglinewithargsret{\sphinxbfcode{\sphinxupquote{populate\_list}}}{}{}
This function currently only works for Windows. It well use the \sphinxstyleemphasis{serial.tools.list\_ports\_windows.comports()}
function to get all available serial ports. Then it will add the names to the list widget.

\end{fulllineitems}

\index{save\_list\_selection() (portsGUI.Ui\_Dialog method)@\spxentry{save\_list\_selection()}\spxextra{portsGUI.Ui\_Dialog method}}

\begin{fulllineitems}
\phantomsection\label{\detokenize{portsGUI:portsGUI.Ui_Dialog.save_list_selection}}\pysiglinewithargsret{\sphinxbfcode{\sphinxupquote{save\_list\_selection}}}{\emph{Dialog}}{}
This function uses the helper function in \sphinxstyleemphasis{settings\_interface} to save the selected serial port when
the user hits the save button.
\begin{quote}\begin{description}
\item[{Parameters}] \leavevmode
\sphinxstyleliteralstrong{\sphinxupquote{Dialog}} \textendash{} This must be of type PyQt5.QtWidgets.QDialog.

\end{description}\end{quote}

\end{fulllineitems}

\index{reset() (portsGUI.Ui\_Dialog method)@\spxentry{reset()}\spxextra{portsGUI.Ui\_Dialog method}}

\begin{fulllineitems}
\phantomsection\label{\detokenize{portsGUI:portsGUI.Ui_Dialog.reset}}\pysiglinewithargsret{\sphinxbfcode{\sphinxupquote{reset}}}{}{}
This function clears the serial port selection.

\end{fulllineitems}


\end{fulllineitems}



\section{sampRateGUI module}
\label{\detokenize{sampRateGUI:module-sampRateGUI}}\label{\detokenize{sampRateGUI:samprategui-module}}\label{\detokenize{sampRateGUI::doc}}\index{sampRateGUI (module)@\spxentry{sampRateGUI}\spxextra{module}}\begin{quote}\begin{description}
\item[{platform}] \leavevmode
Unix, Windows

\item[{synopsis}] \leavevmode
This module is mostly generated code from QtDesigner. It describes the format of the settings
dialog for editing the sample rate of the serial connection to the ADC.

\item[{moduleauthor}] \leavevmode
Michael Eller \textless{}\sphinxhref{mailto:mbe9a@virginia.edu}{mbe9a@virginia.edu}\textgreater{}

\end{description}\end{quote}
\index{Ui\_Dialog (class in sampRateGUI)@\spxentry{Ui\_Dialog}\spxextra{class in sampRateGUI}}

\begin{fulllineitems}
\phantomsection\label{\detokenize{sampRateGUI:sampRateGUI.Ui_Dialog}}\pysigline{\sphinxbfcode{\sphinxupquote{class }}\sphinxcode{\sphinxupquote{sampRateGUI.}}\sphinxbfcode{\sphinxupquote{Ui\_Dialog}}}
Bases: \sphinxcode{\sphinxupquote{object}}

The graphical structure of this \sphinxstyleemphasis{Ui\_Dialog} was generated by QtDesigner.
The window allows the user to set the sample rate of the ADC.
\index{setupUi() (sampRateGUI.Ui\_Dialog method)@\spxentry{setupUi()}\spxextra{sampRateGUI.Ui\_Dialog method}}

\begin{fulllineitems}
\phantomsection\label{\detokenize{sampRateGUI:sampRateGUI.Ui_Dialog.setupUi}}\pysiglinewithargsret{\sphinxbfcode{\sphinxupquote{setupUi}}}{\emph{Dialog}}{}
This function initializes the window by altering the \sphinxstyleemphasis{Dialog} object passed (by reference) to it.
\begin{quote}\begin{description}
\item[{Parameters}] \leavevmode
\sphinxstyleliteralstrong{\sphinxupquote{Dialog}} \textendash{} This must be of type PyQt5.QtWidgets.QDialog.

\end{description}\end{quote}

\end{fulllineitems}

\index{retranslateUi() (sampRateGUI.Ui\_Dialog method)@\spxentry{retranslateUi()}\spxextra{sampRateGUI.Ui\_Dialog method}}

\begin{fulllineitems}
\phantomsection\label{\detokenize{sampRateGUI:sampRateGUI.Ui_Dialog.retranslateUi}}\pysiglinewithargsret{\sphinxbfcode{\sphinxupquote{retranslateUi}}}{\emph{Dialog}}{}
This function translates the GUI objects of this class to the passed \sphinxstyleemphasis{Dialog} objcect,
then sets the text of the various objects in the window.
\begin{quote}\begin{description}
\item[{Parameters}] \leavevmode
\sphinxstyleliteralstrong{\sphinxupquote{Dialog}} \textendash{} This must be of type PyQt5.QtWidgets.QDialog.

\end{description}\end{quote}

\end{fulllineitems}

\index{save() (sampRateGUI.Ui\_Dialog method)@\spxentry{save()}\spxextra{sampRateGUI.Ui\_Dialog method}}

\begin{fulllineitems}
\phantomsection\label{\detokenize{sampRateGUI:sampRateGUI.Ui_Dialog.save}}\pysiglinewithargsret{\sphinxbfcode{\sphinxupquote{save}}}{\emph{Dialog}}{}
This function saves the sample rate in the input box and closes the window.
\begin{quote}\begin{description}
\item[{Parameters}] \leavevmode
\sphinxstyleliteralstrong{\sphinxupquote{Dialog}} \textendash{} This must be of type PyQt5.QtWidgets.QDialog.

\end{description}\end{quote}

\end{fulllineitems}


\end{fulllineitems}



\section{scaleAxesGUI module}
\label{\detokenize{scaleAxesGUI:module-scaleAxesGUI}}\label{\detokenize{scaleAxesGUI:scaleaxesgui-module}}\label{\detokenize{scaleAxesGUI::doc}}\index{scaleAxesGUI (module)@\spxentry{scaleAxesGUI}\spxextra{module}}\begin{quote}\begin{description}
\item[{platform}] \leavevmode
Unix, Windows

\item[{synopsis}] \leavevmode
This module is mostly generated code from QtDesigner. It describes the format of the settings
dialog for editing the limits of the plot axes.

\item[{moduleauthor}] \leavevmode
Michael Eller \textless{}\sphinxhref{mailto:mbe9a@virginia.edu}{mbe9a@virginia.edu}\textgreater{}

\end{description}\end{quote}
\index{Ui\_Dialog (class in scaleAxesGUI)@\spxentry{Ui\_Dialog}\spxextra{class in scaleAxesGUI}}

\begin{fulllineitems}
\phantomsection\label{\detokenize{scaleAxesGUI:scaleAxesGUI.Ui_Dialog}}\pysigline{\sphinxbfcode{\sphinxupquote{class }}\sphinxcode{\sphinxupquote{scaleAxesGUI.}}\sphinxbfcode{\sphinxupquote{Ui\_Dialog}}}
Bases: \sphinxcode{\sphinxupquote{object}}

The graphical structure of this \sphinxstyleemphasis{Ui\_Dialog} was generated by QtDesigner.
The window allows the user to set the limits of the plot axes.
\index{setupUi() (scaleAxesGUI.Ui\_Dialog method)@\spxentry{setupUi()}\spxextra{scaleAxesGUI.Ui\_Dialog method}}

\begin{fulllineitems}
\phantomsection\label{\detokenize{scaleAxesGUI:scaleAxesGUI.Ui_Dialog.setupUi}}\pysiglinewithargsret{\sphinxbfcode{\sphinxupquote{setupUi}}}{\emph{Dialog}}{}
This function initializes the window by altering the \sphinxstyleemphasis{Dialog} object passed (by reference) to it.
\begin{quote}\begin{description}
\item[{Parameters}] \leavevmode
\sphinxstyleliteralstrong{\sphinxupquote{Dialog}} \textendash{} This must be of type PyQt5.QtWidgets.QDialog.

\end{description}\end{quote}

\end{fulllineitems}

\index{retranslateUi() (scaleAxesGUI.Ui\_Dialog method)@\spxentry{retranslateUi()}\spxextra{scaleAxesGUI.Ui\_Dialog method}}

\begin{fulllineitems}
\phantomsection\label{\detokenize{scaleAxesGUI:scaleAxesGUI.Ui_Dialog.retranslateUi}}\pysiglinewithargsret{\sphinxbfcode{\sphinxupquote{retranslateUi}}}{\emph{Dialog}}{}
This function translates the GUI objects of this class to the passed \sphinxstyleemphasis{Dialog} objcect,
then sets the text of the various objects in the window.
\begin{quote}\begin{description}
\item[{Parameters}] \leavevmode
\sphinxstyleliteralstrong{\sphinxupquote{Dialog}} \textendash{} This must be of type PyQt5.QtWidgets.QDialog.

\end{description}\end{quote}

\end{fulllineitems}

\index{save() (scaleAxesGUI.Ui\_Dialog method)@\spxentry{save()}\spxextra{scaleAxesGUI.Ui\_Dialog method}}

\begin{fulllineitems}
\phantomsection\label{\detokenize{scaleAxesGUI:scaleAxesGUI.Ui_Dialog.save}}\pysiglinewithargsret{\sphinxbfcode{\sphinxupquote{save}}}{\emph{Dialog}}{}
This function saves the inputs and closes the window.
\begin{quote}\begin{description}
\item[{Parameters}] \leavevmode
\sphinxstyleliteralstrong{\sphinxupquote{Dialog}} \textendash{} This must be of type PyQt5.QtWidgets.QDialog.

\end{description}\end{quote}

\end{fulllineitems}


\end{fulllineitems}



\section{windowFilterGUI module}
\label{\detokenize{windowFilterGUI:module-windowFilterGUI}}\label{\detokenize{windowFilterGUI:windowfiltergui-module}}\label{\detokenize{windowFilterGUI::doc}}\index{windowFilterGUI (module)@\spxentry{windowFilterGUI}\spxextra{module}}\begin{quote}\begin{description}
\item[{platform}] \leavevmode
Unix, Windows

\item[{synopsis}] \leavevmode
This module is mostly generated code from QtDesigner. It describes the format of the settings
dialog for editing the number of samples used in the window filter.

\item[{moduleauthor}] \leavevmode
Michael Eller \textless{}\sphinxhref{mailto:mbe9a@virginia.edu}{mbe9a@virginia.edu}\textgreater{}

\end{description}\end{quote}
\index{Ui\_Dialog (class in windowFilterGUI)@\spxentry{Ui\_Dialog}\spxextra{class in windowFilterGUI}}

\begin{fulllineitems}
\phantomsection\label{\detokenize{windowFilterGUI:windowFilterGUI.Ui_Dialog}}\pysigline{\sphinxbfcode{\sphinxupquote{class }}\sphinxcode{\sphinxupquote{windowFilterGUI.}}\sphinxbfcode{\sphinxupquote{Ui\_Dialog}}}
Bases: \sphinxcode{\sphinxupquote{object}}

The graphical structure of this \sphinxstyleemphasis{Ui\_Dialog} was generated by QtDesigner.
The window allows the user to set the number of samples used in the moving average filter.
\index{setupUi() (windowFilterGUI.Ui\_Dialog method)@\spxentry{setupUi()}\spxextra{windowFilterGUI.Ui\_Dialog method}}

\begin{fulllineitems}
\phantomsection\label{\detokenize{windowFilterGUI:windowFilterGUI.Ui_Dialog.setupUi}}\pysiglinewithargsret{\sphinxbfcode{\sphinxupquote{setupUi}}}{\emph{Dialog}}{}
This function initializes the window by altering the \sphinxstyleemphasis{Dialog} object passed (by reference) to it.
\begin{quote}\begin{description}
\item[{Parameters}] \leavevmode
\sphinxstyleliteralstrong{\sphinxupquote{Dialog}} \textendash{} This must be of type PyQt5.QtWidgets.QDialog.

\end{description}\end{quote}

\end{fulllineitems}

\index{retranslateUi() (windowFilterGUI.Ui\_Dialog method)@\spxentry{retranslateUi()}\spxextra{windowFilterGUI.Ui\_Dialog method}}

\begin{fulllineitems}
\phantomsection\label{\detokenize{windowFilterGUI:windowFilterGUI.Ui_Dialog.retranslateUi}}\pysiglinewithargsret{\sphinxbfcode{\sphinxupquote{retranslateUi}}}{\emph{Dialog}}{}
This function translates the GUI objects of this class to the passed \sphinxstyleemphasis{Dialog} objcect,
then sets the text of the various objects in the window.
\begin{quote}\begin{description}
\item[{Parameters}] \leavevmode
\sphinxstyleliteralstrong{\sphinxupquote{Dialog}} \textendash{} This must be of type PyQt5.QtWidgets.QDialog.

\end{description}\end{quote}

\end{fulllineitems}

\index{save() (windowFilterGUI.Ui\_Dialog method)@\spxentry{save()}\spxextra{windowFilterGUI.Ui\_Dialog method}}

\begin{fulllineitems}
\phantomsection\label{\detokenize{windowFilterGUI:windowFilterGUI.Ui_Dialog.save}}\pysiglinewithargsret{\sphinxbfcode{\sphinxupquote{save}}}{\emph{Dialog}}{}
This function saves the input and closes the window.
\begin{quote}\begin{description}
\item[{Parameters}] \leavevmode
\sphinxstyleliteralstrong{\sphinxupquote{Dialog}} \textendash{} This must be of type PyQt5.QtWidgets.QDialog.

\end{description}\end{quote}

\end{fulllineitems}


\end{fulllineitems}



\section{settings\_interface module}
\label{\detokenize{settings_interface:module-settings_interface}}\label{\detokenize{settings_interface:settings-interface-module}}\label{\detokenize{settings_interface::doc}}\index{settings\_interface (module)@\spxentry{settings\_interface}\spxextra{module}}\begin{quote}\begin{description}
\item[{platform}] \leavevmode
Unix, Windows

\item[{synopsis}] \leavevmode
This module contains all get/set methods for settings.
It’s not particularly interesting but facilitates clean code.

\item[{moduleauthor}] \leavevmode
Michael Eller \textless{}\sphinxhref{mailto:mbe9a@virginia.edu}{mbe9a@virginia.edu}\textgreater{}

\end{description}\end{quote}
\index{fieldnames (in module settings\_interface)@\spxentry{fieldnames}\spxextra{in module settings\_interface}}

\begin{fulllineitems}
\phantomsection\label{\detokenize{settings_interface:settings_interface.fieldnames}}\pysigline{\sphinxcode{\sphinxupquote{settings\_interface.}}\sphinxbfcode{\sphinxupquote{fieldnames}}\sphinxbfcode{\sphinxupquote{ = {[}\textquotesingle{}window\_samples\textquotesingle{}, \textquotesingle{}sample\_rate\textquotesingle{}, \textquotesingle{}x\_axis\_size\textquotesingle{}, \textquotesingle{}y\_axis\_min\textquotesingle{}, \textquotesingle{}y\_axis\_max\textquotesingle{}{]}}}}
static fieldnames for the settings dict / csv file

\end{fulllineitems}

\index{default\_window\_samples (in module settings\_interface)@\spxentry{default\_window\_samples}\spxextra{in module settings\_interface}}

\begin{fulllineitems}
\phantomsection\label{\detokenize{settings_interface:settings_interface.default_window_samples}}\pysigline{\sphinxcode{\sphinxupquote{settings\_interface.}}\sphinxbfcode{\sphinxupquote{default\_window\_samples}}\sphinxbfcode{\sphinxupquote{ = 1}}}
default number of samples to average in a moving average filter

\end{fulllineitems}

\index{default\_sample\_rate (in module settings\_interface)@\spxentry{default\_sample\_rate}\spxextra{in module settings\_interface}}

\begin{fulllineitems}
\phantomsection\label{\detokenize{settings_interface:settings_interface.default_sample_rate}}\pysigline{\sphinxcode{\sphinxupquote{settings\_interface.}}\sphinxbfcode{\sphinxupquote{default\_sample\_rate}}\sphinxbfcode{\sphinxupquote{ = 10}}}
default sample rate in samples / sec

\end{fulllineitems}

\index{default\_x\_axis\_size (in module settings\_interface)@\spxentry{default\_x\_axis\_size}\spxextra{in module settings\_interface}}

\begin{fulllineitems}
\phantomsection\label{\detokenize{settings_interface:settings_interface.default_x_axis_size}}\pysigline{\sphinxcode{\sphinxupquote{settings\_interface.}}\sphinxbfcode{\sphinxupquote{default\_x\_axis\_size}}\sphinxbfcode{\sphinxupquote{ = 30}}}
default x axis range in seconds

\end{fulllineitems}

\index{default\_y\_axis\_min (in module settings\_interface)@\spxentry{default\_y\_axis\_min}\spxextra{in module settings\_interface}}

\begin{fulllineitems}
\phantomsection\label{\detokenize{settings_interface:settings_interface.default_y_axis_min}}\pysigline{\sphinxcode{\sphinxupquote{settings\_interface.}}\sphinxbfcode{\sphinxupquote{default\_y\_axis\_min}}\sphinxbfcode{\sphinxupquote{ = 0}}}
default y\sphinxhyphen{}axis minimum in volts

\end{fulllineitems}

\index{default\_y\_axis\_max (in module settings\_interface)@\spxentry{default\_y\_axis\_max}\spxextra{in module settings\_interface}}

\begin{fulllineitems}
\phantomsection\label{\detokenize{settings_interface:settings_interface.default_y_axis_max}}\pysigline{\sphinxcode{\sphinxupquote{settings\_interface.}}\sphinxbfcode{\sphinxupquote{default\_y\_axis\_max}}\sphinxbfcode{\sphinxupquote{ = 60}}}
default y\sphinxhyphen{}axis maximum in volts

\end{fulllineitems}

\index{m1 (in module settings\_interface)@\spxentry{m1}\spxextra{in module settings\_interface}}

\begin{fulllineitems}
\phantomsection\label{\detokenize{settings_interface:settings_interface.m1}}\pysigline{\sphinxcode{\sphinxupquote{settings\_interface.}}\sphinxbfcode{\sphinxupquote{m1}}\sphinxbfcode{\sphinxupquote{ = 14.3}}}
slope of V\_measured vs. V\_in for V\_in \textless{} 10 V

\end{fulllineitems}

\index{b1 (in module settings\_interface)@\spxentry{b1}\spxextra{in module settings\_interface}}

\begin{fulllineitems}
\phantomsection\label{\detokenize{settings_interface:settings_interface.b1}}\pysigline{\sphinxcode{\sphinxupquote{settings\_interface.}}\sphinxbfcode{\sphinxupquote{b1}}\sphinxbfcode{\sphinxupquote{ = \sphinxhyphen{}1}}}
intercept of V\_measured vs. V\_in for V\_in \textless{} 10 V

\end{fulllineitems}

\index{m2 (in module settings\_interface)@\spxentry{m2}\spxextra{in module settings\_interface}}

\begin{fulllineitems}
\phantomsection\label{\detokenize{settings_interface:settings_interface.m2}}\pysigline{\sphinxcode{\sphinxupquote{settings\_interface.}}\sphinxbfcode{\sphinxupquote{m2}}\sphinxbfcode{\sphinxupquote{ = 14.976}}}
slope of V\_measured vs. V\_in for V\_in \textgreater{} 10 V

\end{fulllineitems}

\index{b2 (in module settings\_interface)@\spxentry{b2}\spxextra{in module settings\_interface}}

\begin{fulllineitems}
\phantomsection\label{\detokenize{settings_interface:settings_interface.b2}}\pysigline{\sphinxcode{\sphinxupquote{settings\_interface.}}\sphinxbfcode{\sphinxupquote{b2}}\sphinxbfcode{\sphinxupquote{ = \sphinxhyphen{}3.7}}}
slope of V\_measured vs. V\_in for V\_in \textgreater{} 10 V

\end{fulllineitems}

\index{generate\_plotting\_configuration\_file() (in module settings\_interface)@\spxentry{generate\_plotting\_configuration\_file()}\spxextra{in module settings\_interface}}

\begin{fulllineitems}
\phantomsection\label{\detokenize{settings_interface:settings_interface.generate_plotting_configuration_file}}\pysiglinewithargsret{\sphinxcode{\sphinxupquote{settings\_interface.}}\sphinxbfcode{\sphinxupquote{generate\_plotting\_configuration\_file}}}{}{}
Will write a new plot configuration (settings) file in the appropriate location.
This will overwrite any existing settings.
This is used when the user restores default settings from within the GUI.

\end{fulllineitems}

\index{read\_plotting\_configuration() (in module settings\_interface)@\spxentry{read\_plotting\_configuration()}\spxextra{in module settings\_interface}}

\begin{fulllineitems}
\phantomsection\label{\detokenize{settings_interface:settings_interface.read_plotting_configuration}}\pysiglinewithargsret{\sphinxcode{\sphinxupquote{settings\_interface.}}\sphinxbfcode{\sphinxupquote{read\_plotting\_configuration}}}{}{}
Read the current settings configuration. If the file doesn’t exist, generate the default.
\begin{quote}\begin{description}
\item[{Returns}] \leavevmode
Dict object containing the GUI’s settings.

\end{description}\end{quote}

\end{fulllineitems}

\index{save\_plotting\_configuration() (in module settings\_interface)@\spxentry{save\_plotting\_configuration()}\spxextra{in module settings\_interface}}

\begin{fulllineitems}
\phantomsection\label{\detokenize{settings_interface:settings_interface.save_plotting_configuration}}\pysiglinewithargsret{\sphinxcode{\sphinxupquote{settings\_interface.}}\sphinxbfcode{\sphinxupquote{save\_plotting\_configuration}}}{\emph{settings}}{}
This function will edit the file based on the dict of settings passed.
\begin{quote}\begin{description}
\item[{Parameters}] \leavevmode
\sphinxstyleliteralstrong{\sphinxupquote{settings}} \textendash{} Dict of settings

\end{description}\end{quote}

\end{fulllineitems}

\index{set\_window\_samples() (in module settings\_interface)@\spxentry{set\_window\_samples()}\spxextra{in module settings\_interface}}

\begin{fulllineitems}
\phantomsection\label{\detokenize{settings_interface:settings_interface.set_window_samples}}\pysiglinewithargsret{\sphinxcode{\sphinxupquote{settings\_interface.}}\sphinxbfcode{\sphinxupquote{set\_window\_samples}}}{\emph{num}}{}
Sets the number of samples to use in the moving average filter.
\begin{quote}\begin{description}
\item[{Parameters}] \leavevmode
\sphinxstyleliteralstrong{\sphinxupquote{num}} \textendash{} number of samples

\item[{Returns}] \leavevmode
bool indicating whether or not the operation was successful

\end{description}\end{quote}

\end{fulllineitems}

\index{get\_window\_samples() (in module settings\_interface)@\spxentry{get\_window\_samples()}\spxextra{in module settings\_interface}}

\begin{fulllineitems}
\phantomsection\label{\detokenize{settings_interface:settings_interface.get_window_samples}}\pysiglinewithargsret{\sphinxcode{\sphinxupquote{settings\_interface.}}\sphinxbfcode{\sphinxupquote{get\_window\_samples}}}{}{}
Read the saved window samples setting from the file.
\begin{quote}\begin{description}
\item[{Returns}] \leavevmode
saved window samples

\end{description}\end{quote}

\end{fulllineitems}

\index{set\_sample\_rate() (in module settings\_interface)@\spxentry{set\_sample\_rate()}\spxextra{in module settings\_interface}}

\begin{fulllineitems}
\phantomsection\label{\detokenize{settings_interface:settings_interface.set_sample_rate}}\pysiglinewithargsret{\sphinxcode{\sphinxupquote{settings\_interface.}}\sphinxbfcode{\sphinxupquote{set\_sample\_rate}}}{\emph{rate}}{}
Set the sample rate and save it in the settings file.
\begin{quote}\begin{description}
\item[{Parameters}] \leavevmode
\sphinxstyleliteralstrong{\sphinxupquote{rate}} \textendash{} ADC sample rate in samples per second. 1 \textless{} rate \textless{} 100.

\item[{Returns}] \leavevmode
bool indicating whether or not the operation was successful.

\end{description}\end{quote}

\end{fulllineitems}

\index{get\_sample\_rate() (in module settings\_interface)@\spxentry{get\_sample\_rate()}\spxextra{in module settings\_interface}}

\begin{fulllineitems}
\phantomsection\label{\detokenize{settings_interface:settings_interface.get_sample_rate}}\pysiglinewithargsret{\sphinxcode{\sphinxupquote{settings\_interface.}}\sphinxbfcode{\sphinxupquote{get\_sample\_rate}}}{}{}
Get the saved ADC sample rate.
\begin{quote}\begin{description}
\item[{Returns}] \leavevmode
the ADC sample rate (float)

\end{description}\end{quote}

\end{fulllineitems}

\index{set\_x\_axis\_size() (in module settings\_interface)@\spxentry{set\_x\_axis\_size()}\spxextra{in module settings\_interface}}

\begin{fulllineitems}
\phantomsection\label{\detokenize{settings_interface:settings_interface.set_x_axis_size}}\pysiglinewithargsret{\sphinxcode{\sphinxupquote{settings\_interface.}}\sphinxbfcode{\sphinxupquote{set\_x\_axis\_size}}}{\emph{size}}{}
Set the total width of the live plot in seconds. The number of samples depends on this and the sample rate.
Total number of samples in the plot is equal to the x\sphinxhyphen{}axis size * sample rate.
\begin{quote}\begin{description}
\item[{Parameters}] \leavevmode
\sphinxstyleliteralstrong{\sphinxupquote{size}} \textendash{} x\sphinxhyphen{}axis size in seconds. 5 \textless{} int(size) \textless{} 3600.

\item[{Returns}] \leavevmode
bool indicating whether or not the operation was successful.

\end{description}\end{quote}

\end{fulllineitems}

\index{get\_x\_axis\_size() (in module settings\_interface)@\spxentry{get\_x\_axis\_size()}\spxextra{in module settings\_interface}}

\begin{fulllineitems}
\phantomsection\label{\detokenize{settings_interface:settings_interface.get_x_axis_size}}\pysiglinewithargsret{\sphinxcode{\sphinxupquote{settings\_interface.}}\sphinxbfcode{\sphinxupquote{get\_x\_axis\_size}}}{}{}
Get the x\sphinxhyphen{}axis length in seconds.
\begin{quote}\begin{description}
\item[{Returns}] \leavevmode
x\sphinxhyphen{}axis size (int)

\end{description}\end{quote}

\end{fulllineitems}

\index{set\_y\_axis\_min() (in module settings\_interface)@\spxentry{set\_y\_axis\_min()}\spxextra{in module settings\_interface}}

\begin{fulllineitems}
\phantomsection\label{\detokenize{settings_interface:settings_interface.set_y_axis_min}}\pysiglinewithargsret{\sphinxcode{\sphinxupquote{settings\_interface.}}\sphinxbfcode{\sphinxupquote{set\_y\_axis\_min}}}{\emph{minimum}}{}
Set the lower limit of the y\sphinxhyphen{}axis data in volts.
\begin{quote}\begin{description}
\item[{Parameters}] \leavevmode
\sphinxstyleliteralstrong{\sphinxupquote{minimum}} \textendash{} lower limit of data y\sphinxhyphen{}axis. \sphinxhyphen{}60 \textless{} minimum \textless{} 69.9

\item[{Returns}] \leavevmode
bool indicating if the operation was so successful or not

\end{description}\end{quote}

\end{fulllineitems}

\index{get\_y\_axis\_min() (in module settings\_interface)@\spxentry{get\_y\_axis\_min()}\spxextra{in module settings\_interface}}

\begin{fulllineitems}
\phantomsection\label{\detokenize{settings_interface:settings_interface.get_y_axis_min}}\pysiglinewithargsret{\sphinxcode{\sphinxupquote{settings\_interface.}}\sphinxbfcode{\sphinxupquote{get\_y\_axis\_min}}}{}{}
Get the y\sphinxhyphen{}axis lower limit in volts.
\begin{quote}\begin{description}
\item[{Returns}] \leavevmode
y\sphinxhyphen{}axis minimum (float)

\end{description}\end{quote}

\end{fulllineitems}

\index{set\_y\_axis\_max() (in module settings\_interface)@\spxentry{set\_y\_axis\_max()}\spxextra{in module settings\_interface}}

\begin{fulllineitems}
\phantomsection\label{\detokenize{settings_interface:settings_interface.set_y_axis_max}}\pysiglinewithargsret{\sphinxcode{\sphinxupquote{settings\_interface.}}\sphinxbfcode{\sphinxupquote{set\_y\_axis\_max}}}{\emph{maximum}}{}
Set the upper limit of the y\sphinxhyphen{}axis in volts. 1 \textless{} maximum \textless{} 70
\begin{quote}\begin{description}
\item[{Parameters}] \leavevmode
\sphinxstyleliteralstrong{\sphinxupquote{maximum}} \textendash{} upper limit of the y\sphinxhyphen{}axis in volts

\item[{Returns}] \leavevmode
bool indicating whether or not the operation was successful

\end{description}\end{quote}

\end{fulllineitems}

\index{get\_y\_axis\_max() (in module settings\_interface)@\spxentry{get\_y\_axis\_max()}\spxextra{in module settings\_interface}}

\begin{fulllineitems}
\phantomsection\label{\detokenize{settings_interface:settings_interface.get_y_axis_max}}\pysiglinewithargsret{\sphinxcode{\sphinxupquote{settings\_interface.}}\sphinxbfcode{\sphinxupquote{get\_y\_axis\_max}}}{}{}
Get the y\sphinxhyphen{}axis upper limit in volts.
\begin{quote}\begin{description}
\item[{Returns}] \leavevmode
the y\sphinxhyphen{}axis maximum (float)

\end{description}\end{quote}

\end{fulllineitems}

\index{save\_port\_configuration() (in module settings\_interface)@\spxentry{save\_port\_configuration()}\spxextra{in module settings\_interface}}

\begin{fulllineitems}
\phantomsection\label{\detokenize{settings_interface:settings_interface.save_port_configuration}}\pysiglinewithargsret{\sphinxcode{\sphinxupquote{settings\_interface.}}\sphinxbfcode{\sphinxupquote{save\_port\_configuration}}}{\emph{port}}{}
This function sets the separate file that indicates which serial port to use.
\begin{quote}\begin{description}
\item[{Parameters}] \leavevmode
\sphinxstyleliteralstrong{\sphinxupquote{port}} \textendash{} ADC serial port (string)

\end{description}\end{quote}

\end{fulllineitems}

\index{read\_port\_configuration() (in module settings\_interface)@\spxentry{read\_port\_configuration()}\spxextra{in module settings\_interface}}

\begin{fulllineitems}
\phantomsection\label{\detokenize{settings_interface:settings_interface.read_port_configuration}}\pysiglinewithargsret{\sphinxcode{\sphinxupquote{settings\_interface.}}\sphinxbfcode{\sphinxupquote{read\_port\_configuration}}}{}{}
Get the saved ADC serial port.
\begin{quote}\begin{description}
\item[{Returns}] \leavevmode
serial port (string)

\end{description}\end{quote}

\end{fulllineitems}

\index{restore\_defaults() (in module settings\_interface)@\spxentry{restore\_defaults()}\spxextra{in module settings\_interface}}

\begin{fulllineitems}
\phantomsection\label{\detokenize{settings_interface:settings_interface.restore_defaults}}\pysiglinewithargsret{\sphinxcode{\sphinxupquote{settings\_interface.}}\sphinxbfcode{\sphinxupquote{restore\_defaults}}}{}{}
This function resets all settings to the defaults stored in this file.

\end{fulllineitems}



\renewcommand{\indexname}{Python Module Index}
\begin{sphinxtheindex}
\let\bigletter\sphinxstyleindexlettergroup
\bigletter{a}
\item\relax\sphinxstyleindexentry{animation}\sphinxstyleindexpageref{animation:\detokenize{module-animation}}
\indexspace
\bigletter{m}
\item\relax\sphinxstyleindexentry{mainWindow}\sphinxstyleindexpageref{mainWindow:\detokenize{module-mainWindow}}
\indexspace
\bigletter{p}
\item\relax\sphinxstyleindexentry{plotGUI}\sphinxstyleindexpageref{plotGUI:\detokenize{module-plotGUI}}
\item\relax\sphinxstyleindexentry{portsGUI}\sphinxstyleindexpageref{portsGUI:\detokenize{module-portsGUI}}
\indexspace
\bigletter{s}
\item\relax\sphinxstyleindexentry{sampRateGUI}\sphinxstyleindexpageref{sampRateGUI:\detokenize{module-sampRateGUI}}
\item\relax\sphinxstyleindexentry{scaleAxesGUI}\sphinxstyleindexpageref{scaleAxesGUI:\detokenize{module-scaleAxesGUI}}
\item\relax\sphinxstyleindexentry{settings\_interface}\sphinxstyleindexpageref{settings_interface:\detokenize{module-settings_interface}}
\indexspace
\bigletter{w}
\item\relax\sphinxstyleindexentry{windowFilterGUI}\sphinxstyleindexpageref{windowFilterGUI:\detokenize{module-windowFilterGUI}}
\end{sphinxtheindex}

\renewcommand{\indexname}{Index}
\printindex
\end{document}